\documentclass{article}
\usepackage{mathbbol}
\usepackage{graphicx}

\graphicspath{ {./images/} }
\renewcommand{\arraystretch}{2.5}
\newcommand{\definition}[1]{\paragraph*{\underline{Definition}: #1}}
\newcommand{\property}[1]{\paragraph*{\underline{Property}: #1}}
\renewcommand{\vec}[1]{\overrightarrow{#1}}
\newcommand{\scmul}[2]{\vec{#1} \cdot \vec{#2}}
\newcommand{\twodivec}[3]{\vec{#1} \begin{pmatrix}#2 \\ #3\end{pmatrix}}

\title{Real Random Variables}
\author{Flydexo}

\begin{document}
\maketitle
\tableofcontents
\section{Real random variables}
We consider a random experience of universe $\Omega = \{e_1;e_2;e_3;...;e_r\}$ is finite and a law of probability $p$ over $\Omega$
\definition{Real random variable (discrete)}
An RRV $X$ over $\Omega$ is a function that associates a real for each issue of $\Omega$. We note $\{X=a\}$ the event $X$ taking the value $a$ and $p(X=a)$ its probability.
\definition{}
Let X a RRV over $\Omega$ with the values $x_1,x_2,...,x_n$. When each value $x_i$, we associate the probability $p_i = p(X=x_i)$ we define the law of probability of X.
\section{Expectation - Variance - Standard gap}
\subsection{Definitions}
\definition{Expectation}
Expectation of X is the Real noted $E(X)$ defined by: $$E(X) = p_1x_1+p_2x_2+...+p_nx_n$$
\definition{Variance}
Variance of X noted $V(X)$ defined by: $$V(X) = p_1(x_1-E(X))^2 + p_2(x_2-E(X))^2 + ... + p_n(x_n-E(X))^2$$
\definition{Standard gap}
The standard gap of X is the real noted $\sigma(X)$ defined by: $$\sigma(X) = \sqrt{V(X)}$$
\subsection{Properties of the indicators}
\property{Formula of König-Huygens}
$$V(X) = p_1(x_1)^2 + p_2(x_2)^2 + ... + p_n(x_n)^2 - (E(X))^2$$
\definition{Random variable aX+b}
For every real a and b, we can associate a new random variable by associating each issue giving the value $x_i$, the real $ax_i+b$. Named $aX+b$
\property{E(aX+b) and V(aX+b)}
Let a and b be two reals. We have:
\begin{itemize}
    \item $E(aX+b) = aE(X)+b$
    \item $V(aX+b) = a^2V(X)$
    \item $\sigma(aX+b) = |a|\sigma(X)$
\end{itemize}
\property{Expectation and simulation}
With a sufficiently big sample of values taken by a random variable, the average of its values is close to the value of the expectation of this random variable.
\end{document}